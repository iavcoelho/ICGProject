\documentclass{report}
\title{ISS Docking Simulator}
\author{Igor Coelho - 113532}
\usepackage{hyperref}
\date{\today}
\begin{document}
\maketitle
\section{URLs}
Github Repository: \url{https://github.com/iavcoelho/ICGProject}

Live Project: \url{https://iavcoelho.github.io/ICGProject}
\section{Overview}
The objective of this project is to make a game similar to the ISS
Docking Simulator (https://iss-sim.spacex.com/), developed by SpaceX
and NASA.
In this game, there is a capsule that spawns at a random distance from
the ISS, having equally random (within a defined interval) pitch, roll and
yaw angles. The objective of the player is to correctly align the capsule
with the ISS, in order to successfully dock.
The interaction between the player can be made with keyboard, where
different keys control the velocity of the capsule in a specific direction
(For example, using de WASD keys to move up, down, left, right and the
Q and E keys to accelerate/decelerate towards the ISS).
The capsule will be able to move/rotate along all 3 axis
\section{Objectives}
Creation of a game similar to the one present in https://iss-sim.spacex.com/
\begin{itemize}
\item Have the International Space Station in Space
\item Player is located on a separate module
\item Player must control the module in order to dock on the ISS
\item All axis of movement and rotation can be controlled
\item Position,  rotation and absolute velocity must be below a threshold (to be successful)
\end{itemize}
\section{What is done}

Currently the models have been added to the scene, the camera has been anchored to the module and the module moves up.
There have also been placed BufferGeometries to simulate stars

\section{What is missing}

The movement of the module needs to be controlled by the player, which must be able to do so by a widget and by using the keyboard.
There needs to be some widget that displays the current rotation, position and velocity values.
Texture and environment must be added.
The first person camera must also be added, for the user to have a 1st person view from the module.

\end{document}
